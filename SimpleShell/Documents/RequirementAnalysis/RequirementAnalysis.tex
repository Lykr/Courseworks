\documentclass{article}

\pagestyle{headings}

\title{REQUIREMENT ANALYSIS}
\author{WANG Chen, 2015213086, wangchen@bupt.edu.cn}
\date{\today}

\begin{document}

\maketitle

\tableofcontents

\newpage
\section{INTRODUCTION}
This is a requirement analysis document for a simple shell program named msh.
The shell program is a course assignment in \emph{Advanced Programming in the UNIX Environment}.
It will make us have a deeper understanding in UNIX C programming, after we finish this program.
A summary of requirements of the shell program will be presented below.
And detailed implementation of these requirements are presented in DETAILED DESIGN document.

\newpage
\section{FUNCTIONAL REQUIREMENTS}

\subsection{Command Prompt}
\subsubsection{Description}
The program is executed from the console and a command prompt is displayed when it is started. For example, "\verb|->|". Users can change the command prompt by assigning values to specific environment variables.

\subsubsection{Case}
\begin{enumerate}
\item User input \verb|./msh| to run the shell program.
\item The terminal will present a \verb|->| to indicate that the shell program is running.
\item User close the shell.
\item User input \verb|export "msh_cp=<new_command_prompt>"|.
\item User input \verb|./msh| to run the shell.
\item \verb|new_command_prompt| will be present.
\end{enumerate}

\subsection{Close Program}
\subsubsection{Description}
This program can be shut down normally by a special command or key combination.

\subsubsection{Case}
\begin{enumerate}
\item User input \verb|./msh| to run the shell program.
\item User input \verb|Ctrl + C|, \verb|Ctrl + D| or \verb|exit| to shut down the shell.
\item User return to the terminal
\end{enumerate}

\subsection{Runs in Background}
\subsubsection{Description}
Provide background operation mechanism. Tasks submitted by users can be made to run in the background by some instructions, such as: \verb|-> bg job1 <CR>| which will make \verb|job1| run in the background and return a new prompt to the user immediately.

\subsubsection{Case}
\begin{enumerate}
\item User input \verb|./msh| to run the shell program.
\item User input \verb|command &| to run this command in background.
\end{enumerate}

\subsection{Output \& Input Redirection}
\subsubsection{Description}
Write all output of task to a file by specifying file names instead of sending them to standard output.
By specifying the file name, the task can obtain the required data from the corresponding file, rather than from the standard input.

\subsubsection{Output Redirection Case}
\begin{enumerate}
\item User input \verb|./msh| to run the shell program.
\item User input \verb|command > cmd_out|
\item A file named \verb|cmd_out| will be created and its content contains the output of the command.
\end{enumerate}

\subsubsection{Input Redirection Case}
\begin{enumerate}
\item User input \verb|./msh| to run the shell program.
\item User input \verb|command < cmd_input|
\item The command will take the content of \verb|cmd_input| as input.
\end{enumerate}

\subsubsection{Output \& Input Redirection Case}
\begin{enumerate}
\item User input \verb|./msh| to run the shell program.
\item User input \verb|command < cmd_input > cmd_output|.
\item The command will take the content of \verb|cmd_input| as input, and write output to the file named \verb|cmd_output|.
\end{enumerate}

\subsection{Other Command}
\subsubsection{Description}
The shell can execute some command like \verb|cd|, \verb|ls|, \verb|mkdir|, \verb|rm| and so on.

\subsubsection{Case}
\begin{enumerate}
\item User input \verb|./msh| to run the shell program.
\item User input \verb|ls|.
\item The list of file and dir in current dir will be presented.
\end{enumerate}

\newpage
\section{NON-FUNCTIONAL REQUIREMENTS}

\begin{itemize}
\item The program code is easy to read.
\item The program is robust.
\end{itemize}

\newpage
\section{DEVELOPMENT ENVIRONMENT}

\begin{itemize}
\item Hardware: Macbook pro 2018
\item System: Mac OS 10.14.4
\item Code Software: Visual Studio Code
\item Compiler: GCC 4.2.1
\end{itemize}

\end{document}