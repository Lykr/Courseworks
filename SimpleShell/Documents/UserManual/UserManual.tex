\documentclass{article}

\pagestyle{headings}

\title{USER MANUAL}
\author{WANG Chen, 2015213086, wangchen@bupt.edu.cn}
\date{\today}

\begin{document}

\maketitle

\tableofcontents

\newpage
\section{INTRODUCTION}

This is a user manual for simple shell program named msh.
The shell program is a course assignment in \emph{Advanced Programming in the UNIX Environment}.
It will make us have a deeper understanding in UNIX C programming, after we finish this program.
This document introduce the shell program and illustrate how to use the shell program.

\section{HOW TO USE}

\subsection{Start the shell}

\begin{enumerate}
\item Open the terminal of system.
\item Get to the folder of \verb|msh|.
\item Enter \verb|./msh| to start program.
\end{enumerate}

\subsection{Run Command}

After you start msh, you can use it as normal shell program.
msh can run most of command, like \verb|cd|, \verb|ls| and so on.
You can enter \verb|help| to get more information about msh, and built-in functions.

\subsection{Run Command in Background}

If you want to run a command in background, then enter an other command immediately.
You can append a \verb|&| to command for doing that.
For example, \verb|./msh &| will run a new msh in background.

\subsection{Input \& Output Redirection}

The shell can implement input and output redirection by using \verb|<| and \verb|>| symbol.
User can use \verb|command > <output_file>| to write the result of \verb|command| to \verb|<output_file>|.
Or you can use \verb|command < <input_file>| to take \verb|<input_file>| as input of the \verb|command|.
And you can even combine them like, \verb|command| \verb|<| \verb|<input_file>| \verb|>| \verb|<output_file>| to take \verb|<input_file>| as input and write output to the \verb|<output_file>|.

\subsection{Change Command Prompt}

User can modify command prompt when the shell is closing.

\begin{enumerate}
\item Close the shell program, if it is running.
\item use command \verb|export "msh_cp=<new command prompt>"| to change the command prompt.
\item Enter \verb|./msh| to start program, and <new command prompt> will be presented.
\end{enumerate}

\subsection{Close Program}

User can close the shell by entering \verb|exit| command or \verb|Ctrl + C| or \verb|Ctrl + D| key combination.

\section{FAQ}

\begin{itemize}
\item Why some command like \verb|jobs| can not run in this shell program?

This program is a simple implementation of shell.
It is not implement all functions which other shell program have.

\item How to make the program running in background return to foreground?

There is not way to do this.
So you best not to run some command which cannot finished in background.
Otherwise, you need to use ``Resources Manager'' to shut down the program you have run.

\end{itemize}

\newpage
\section{CONTACT INFORMATION}

\begin{itemize}
\item Author: WANG Chen
\item Location: Beijing University of Posts and Telecommunications No.10 Xitucheng Road, Haidian District, Beijing, China.
\item Email: wangchen@bupt.edu.cn
\item Telephone no.: 15011090XXX
\end{itemize}

\end{document}